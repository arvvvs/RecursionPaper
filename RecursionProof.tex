\documentclass{article}
\usepackage{verbatim}
\usepackage{amsfonts}
\usepackage{amssymb}
\usepackage{mathtools}
\newcommand{\itab}[1]{\hspace{0em}\rlap{#1}}
\newcommand{\tab}[1]{\hspace{.2\textwidth}\rlap{#1}}
\begin{document}

\title{Recursion and Applications}
\author{Akshay Raj Verma}
\maketitle
{Definition.} A method of defining functions in which the function is being defined is applied by it's own definition
\\
Methods that exhibit recursive behavior can be defined by two properties: \\
\begin{enumerate}
	\item A base case (or cases)
	\item An algorithm which is used to solve instances of smaller problems, using the solution and the algorithm to solve the original problem
\end{enumerate}
{\bf Example:} The Fibonacci Sequence is a famous example of a recursive function\\
\hspace{10mm}{\bf Base Case:}  \hspace{10mm}$f_0=0$ and $f_1=1$
\\
\hspace{10mm}{\bf Formula:} \hspace{11mm} $f(n)=f(n-1)+f(n-2)$ for all $\mathbb{N}$ $ n \geq 2$
\\
{\bf Implementation:}
\\
\begin{center}
$f(0)=0$
\\
$f(1)=1$
\\
$f(2)=f(2-1)+f(2-2)=f(1)+f(0)=1$
\\
$f(3)=f(3-1)+f(3-2)=f(2)+f(1)=2$
\\
$f(4)=f(4-1)+f(4-2)=f(3)+f(2)=3$
\\
$f(5)=f(5-1)+f(5-2)=f(4)+f(3)=5$
\\
$...$
\\
\end{center}
As shown by the above sequences solving $f(2)$ requires the solution for $f(1)$ and $f(0)$.  Once you have the solution to both of them, you apply the Fibonnaci Formula to get $f(2)$.  This process is again repeated for $f(3)$ which requires the solutions to $f(2)$ which in turn as noted above requires $f(1)$ and $f(0)$.  The Fibonacci Formula is again used to solve it, and so on.  
\\
We can see the use of recursion in the Fibonacci Sequence in the procedure when in order to solve the Fibonacci Sequence for $n$ when we invoke the procedure in order to solve the problem itself.   
\\
In conclusion in order to solve the Fibonacci Sequence for f(n+2) using the recursive algorithm above is not possible without first knowing the solutions for f(n+1) and f(n), and calls the Fibonacci formula $n+1$ times to be solved.  
\\
However the nth sequence of the Fibonacci Sequence can also be produced by Binet's formula.
\\
\\
{\bf Theorem:} For all n in $\mathbb{N}$ the Fibonacci Number, we denote $F_n$, can be given directly by the Binet Formula($B_n$) as given below:
\\
\begin{center}
	$B_n=\frac{1}{\sqrt[]{5}}((\frac{1+\sqrt[]{5}}{2})^n-(\frac{1-\sqrt[]{5}}{2})^n)$
\end{center}
~\\
{\bf Proof:} We will prove Binet's Formula through the use of strong induction.
\\
{\bf Lemma 1.} 
If $B_n=F_n$, then for the base cases n=0, 1 $F_0=B_0$ and $F_1=B_1$  
\\
{\bf Proof.}
\\
	We want to show that $B_0=F_0$ and $B_1=F_1$
	\\
	For base case $n=0$:
	\\
	First we substitute $0$ for $n$ $B_0=\frac{1}{\sqrt[]{5}}((\frac{1+\sqrt[]{5}}{2})^0-(\frac{1-\sqrt[]{5}}{2})^0)$
	\\
	Which is equal to $\frac{1}{\sqrt[]{5}}((1)-(1))$
	\\
	Therefore $B_0=0$
	\\
	And so by the Transitive Property $B_0=F_0$
	\\
	For base case $n=1$:
	\\
	First we substitute $1$ for $n$ $B_1=\frac{1}{\sqrt[]{5}}((\frac{1+\sqrt[]{5}}{2})^1-(\frac{1-\sqrt[]{5}}{2})^1)$
	\\
	So $\frac{1}{\sqrt[]{5}}((\frac{1+\sqrt[]{5}}{2})-(\frac{1-\sqrt[]{5}}{2}))$
	\\
	Which is equal to $\frac{1}{\sqrt[]{5}}(\frac{(1+\sqrt[]{5})-(1-\sqrt[]{5})}{2})$
	\\
	Therefore $=\frac{1}{\sqrt[]{5}}(\frac{2*\sqrt[]{5}}{2})$
	\\
	Simplifying further $=\frac{1}{\sqrt[]{5}}(\sqrt[]{5})$
	\\
	Therefore $B_1=1$ 
	\\
	And so by the Transitive Propert $B_1=F_1$


	For Base Cases $n=0$ and $n=1$, Binet's Formula is true.  
$\blacksquare$\\
{\bf Lemma 2.}1
If Binet's Formula is true for all $k \in {\mathbb N}$ then it is true for $k+1$, for $k>1$.
\\
{\bf Proof.}
\\
To demonstrate Binet's Formula is true for k+1 we must demonstrate that:
\begin{center}
	$F_{k+1}=B_{k+1}$
	\\
	Which is equal to $\frac{1}{\sqrt[]{5}}((\frac{1+\sqrt[]{5}}{2})^{k+1}-(\frac{1-\sqrt[]{5}}{2})^{k+1})$
\end{center}
We can rewrite Binet's Formula in terms of $\alpha$ and $\beta$.  $\alpha = \frac{1+\sqrt[]{5}}{2}$ and $\beta = \frac{1-\sqrt[]{5}}{2}$. $\alpha$ can be recognized also as the Golden Ratio.   \\
Substituting in $\alpha$ and $\beta$ into Binet's Formula: \begin{center}$B_n=\frac{\alpha^n - \beta^n}{\sqrt[]{5}}$\end{center}
~\\
{\bf Lemma 2.1.} $\alpha+1 = \alpha^{2}$ and $\beta+1=\beta^{2}$ 
\\
{\bf Proof.}
In order to solve Lemma 2 we want to show that $\alpha+1$=$\alpha^2$ and $\beta+1=\beta^{2}$
\begin{center}
	$\alpha+1= \frac{1+\sqrt[]{5}}{2}+1$
	$=\frac{1+\sqrt[]{5}}{2}+\frac{2}{2}$
	$=\frac{3+\sqrt[]{5}}{2}$
	\\
	$\alpha^{2} = (\frac{1+\sqrt[]{5}}{2})^{2}$
	$=\frac{1+2\sqrt[]{5}+5}{4}$
	$=\frac{6+2\sqrt[]{5}}{4}$
	$=\frac{3+\sqrt[]{5}}{2}$
	\\
	~\\
	So by the transitive property $\alpha^2 = \alpha+1$
	~\\
	$\beta+1 = \frac{1-\sqrt[]{5}}{2}+1$
	$=\frac{1-\sqrt[]{5}}{2}+\frac{2}{2}$
	$=\frac{3-\sqrt[]{5}}{2}$
	\\
	$\beta^{2} = (\frac{1+\sqrt[]{5}}{2})^{2}$
	$=\frac{1-2\sqrt[]{5}+5}{4}$
	$=\frac{6-2\sqrt[]{5}}{4}$
	$=\frac{3-\sqrt[]{5}}{2}$
	\\
	~\\
	So by the transitive property $\beta^2 = \beta+1$$\blacksquare$

\end{center}


{\bf Conclusion} Now we continue with the proof of Lemma 2.1, and we proceed using the defintion of Fibonnaci sequence and induction.
\begin{center}
	$F_{k+1}=F_{k}+F_{k-1}$
	\\
	$
	=\frac{1}{\sqrt[]{5}}(\alpha^k-\beta^k)+\frac{1}{\sqrt[]{5}}(\alpha^{k-1}-\beta^{k-1})
$
\\
$
=
\frac{1}{\sqrt[]{5}}(\alpha^k-\beta^k-\beta^k-\beta^{k-1})
$
\\
$=\frac{1}{\sqrt[]{5}}(\alpha^{k-1}(\alpha+1)-\beta^{k-1}(\beta+1))$
\\
$=\frac{1}{\sqrt[]{5}}(\alpha^{k-1}\alpha^{2}-\beta^{k-1}\beta^2)$
\\
Using Lemma 2.1
$=\frac{1}{\sqrt[]{5}}(\alpha^{k+1}-\beta^{k+1}$)
\\
\end{center}
Using the Principle of induction we have shown that $B_n=F_n$ for all $n \in {\mathbb N}$, by proving the Base Cases of n=0 and n=1, then using induction we found that the theory is true for all $n \in {\mathbb N}$ for all $n>1$, by applying the recursive definition of the Fibonacci sequence.   
$\blacksquare$
\\
~\\
~\\
{\bf Conclusions:} While both formulas for the Fibonacci series give the same resultsfor $n$, from the defintion of computational mathematics they are very different. To find Fibinocci series recursively using $F_n$ requires $n-1$ sums or $O(n)$ time in big O notation. To computationally find the solution to $n$ using Binet's Formula would actually takes longer in cases where $n$ is a relatively small number due to the relative complexity of Binet's Formula.  However to find the Fibonnaci Number for a large $n$ it is faster to use Binet's Formula. The recursive defintion for the Fibonnaci Sequence however is much more efficient if one wishes to find the Fibonnaci Numbers in traversal order from $1$ to $n$.      
\\
\begin{comment}
	ToDo: Factorials, Other math stuff (like Catlan numbers) glanced over, Finally Computer Science implementation,Other applications: in nature Fractals
	also: recursive humor
\end{comment}
\\
{\bf Example} Factorials are another example of a recursive function defined as $n!$ for all nonnegative $\mathbb{Z}$
\\
{\bf Base Case} $n=1$, where the factoral of$1 (denoted 1!)$, is$ 1$
\\
{\bf Formula} $n! = n * (n-1)!$ for all non negative $\mathbb{Z}$ 
\\
{\bf Implementation} \\
\begin{center}
	1!=1
	\\
	2!=2*1!=2
	\\
	3!=3*2!=6
	\\
	4!=4*3!=24
	\\
	5!=5*4!=120
	\\
	6!=6*5!=720
	\\
	...

\end{center}
~\\
Factorials can be defined as the product of all positive $\mathbb{Z}$ less than $n$.  


\end{document}

